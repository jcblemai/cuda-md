\documentclass[a4wide,12pt]{article}
\usepackage{a4wide}
\usepackage[utf8]{inputenc}
\usepackage[english]{babel} 
\usepackage{url}

\usepackage{color}
\usepackage{graphicx}
\usepackage{fancyvrb}
\usepackage{amsmath}

\date{\today}


\title{\textsc{GPU parallelization of a Molecular Dynamics Simulation}}

\author{\textsc{Joseph Lemaitre}}

\begin{document}

\maketitle

\begin{abstract}
Molecular Dynamic (MD) simulation algorithm are known for being highly parallelization responsive~\cite{walters}.
We aim to verify this affirmation by doing the parallelization of an 
external C++ code~\cite{ljfluid} written by 
Wolfgang Wieser. It's a MD simulation of a Lennard-Jones fluid. It provides
a straight-forward approach to the problem, with an quasi-direct implementation of the 
algorithm. It also comes with gnuplot-ready data logging allowing correctness verification.

We will use NVIDIA's parallel computing architecture : CUDA.

\end{abstract}

\section{Presentation of the Problem}
I will talk about the integration-method, the cut-off radius, the boundary conditions...

\section{Parallelization}
\subsection{Goal}
I will explain why it's a slow algorithm, the costly steps and theirs complexity (force evaluation, time integration, ...) with maybe a profiling on 
the single thread version. And then why CUDA, and why a speed-up is expected, talk about the fact that MD is non memory intensive...
\subsection{Methods}
There is several methods~\cite{plimpton} to parrallelize MD simulation. I will present them here and choose 
one (or if I have time, several for comparison).



\section{Implementation Details}
All the remarks about the code, the problems encountered and their solutions
will go here. I will write this section with the code.


\section{Execution and Results}
\subsection{Execution Setup}
I will give information about
\begin{itemize}
    \item The simulation data (Number of atoms, Time-step, Cut-off radius...) for all the runs.
    \item The hardware setup.
    \item Software information (versions, flags...)
    \item The bench-marking methodology.
    \item Concerns about the accuracy and extensiveness of the evaluation.
\end{itemize}

\subsection{Results}
Include graphics of speed-up in respect to the simulation data, discussions, possible amelioration. If I had the time
to test several methods, I'll include a comparison.



\begin{thebibliography}{10}

\bibitem{ljfluid}
Wolfgang Wieser.
\newblock {{\it Simple molecular dynamics simulation of a Lennard-Jones fluid}},
\newblock \url{http://www.triplespark.net/sim/ljfluid/}.

\bibitem{walters}
John Paul Walters, Vidyananth Balu, 
Vipin Chaudhary, David Kofke, Andrew Schultz.
\newblock {\it Accelerating Molecular Dynamics Simulations with GPUs}.
\newblock University at Buffalo, Buffalo, NY 14260.

\bibitem{plimpton}
Steve Plimpton.
\newblock {\it Fast Parallel Algorithms for Short–Range Molecular Dynamics},
\newblock in the {\em Journal of Computational Physics}, vol 117, pages 1--19, 1995.




\end{thebibliography}


\end{document}
